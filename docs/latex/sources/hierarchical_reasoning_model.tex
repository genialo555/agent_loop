\documentclass[11pt,twocolumn]{article}
\usepackage[utf8]{inputenc}
\usepackage[T1]{fontenc}
\usepackage{amsmath,amsfonts,amssymb}
\usepackage{graphicx}
\usepackage{booktabs}
\usepackage{algorithm}
\usepackage{algorithmic}
\usepackage{natbib}
\usepackage{url}
\usepackage{hyperref}
\usepackage{geometry}
\usepackage{tikz}
\usepackage{pgfplots}

% Page geometry
\geometry{margin=1in}

% Title and authors
\title{Hierarchical Reasoning Model}
\author{Guan Wang$^{1,\dagger}$, Jin Li$^1$, Yuhao Sun$^1$, Xing Chen$^1$, Changling Liu$^1$, \
        Yue Wu$^1$, Meng Lu$^{1,\dagger}$, Sen Song$^{2,\dagger}$, Yasin Abbasi Yadkori$^{1,\dagger}$ \
        $^1$Sapient Intelligence, Singapore \
        $^2$Tsinghua University}

\date{}

\begin{document}

\maketitle

\begin{abstract}
Reasoning, the process of devising and executing complex goal-oriented action sequences,
remains a critical challenge in AI. Current large language models (LLMs) primarily employ
Chain-of-Thought (CoT) techniques, which suffer from brittle task decomposition, extensive
data requirements, and high latency. Inspired by the hierarchical and multi-timescale
processing in the human brain, we propose the Hierarchical Reasoning Model (HRM), a novel
recurrent architecture that attains significant computational depth while maintaining both
training stability and efficiency.
\end{abstract}

\section{Introduction}
% TODO: Add introduction content from PDF

\section{Hierarchical Reasoning Model}
% TODO: Add methodology content

\section{Results}
% TODO: Add results and analysis

\section{Brain Correspondence}
% TODO: Add brain correspondence analysis

\section{Related Work}
% TODO: Add related work section

\section{Conclusion}
% TODO: Add conclusion

ibliographystyle{plain}
ibliography{references}

nd{document}
